\section{Paradigma No 03 – Programación Lógico} 

\begin{enumerate}[1.]
	\item La Programación Lógica  
¿Qué es?
La programación lógica es un paradigma que se encuentra dentro del paradigma de la programación funcional. Aunque no es tan conocido como otros paradigmas de programación, es realmente interesante. Se basa en la declaración de hechos y reglas que permiten ir creando lo que para nosotros sería el conocimiento. Aunque inicialmente sea un poco complejo de entender, la programación lógica trabaja de forma muy similar a los humanos en cuanto al manejo de información y conocimientos se refiere. Veamos un ejemplo para entender mejor cómo es la declaración de las reglas y hechos
Paradigma de programación basado en la lógica de primer orden. La Programación Lógica estudia el uso de la lógica para el planteamiento de problemas y el control sobre las reglas de inferencia para alcanzar la solución automática.
La Programación Lógica, junto con la funcional, forma parte de lo que se conoce como Programación Declarativa, es decir la programación consiste en indicar como resolver un problema mediante sentencias, en la Programación Lógica, se trabaja en una forma descriptiva, estableciendo relaciones entre entidades, indicando no como, sino que hacer, entonces se dice que la idea esencial de la Programación Lógica es

	

\end{enumerate}

\section{Paradigma No 04 – Programacion Funcional} 

\begin{enumerate}[1.]
	\item La Programacion Funcional  es una forma en la cual podemos resolver diferentes problemáticas ya que estaremos trabajando principalmente con funciones, evitaremos los datos mutables, así como el hecho de compartir estados entre funciones.
	\\Las funciones serán tratadas como ciudadanos de primera clase. Las funciones podrán ser asignadas a variables además podrán ser utilizadas como entrada y salida de otras funciones.
	

\end{enumerate}
